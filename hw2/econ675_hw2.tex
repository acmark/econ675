\documentclass[12pt]{article}
 \usepackage[margin=1in]{geometry}
\usepackage{amsmath,amsthm,mathtools, amssymb,amsfonts,listings,color,graphicx}

\newcommand{\N}{\mathbb{N}}
\newcommand{\Z}{\mathbb{Z}}
\newcommand{\R}{\mathbb{R}}
\newcommand{\E}{\mathbb{E}}
\newcommand{\V}{\mathbb{V}}

\newcommand{\vpar}{\vspace{.3cm}}

\definecolor{mygreen}{RGB}{28,172,0} % color values Red, Green, Blue
\definecolor{mylilas}{RGB}{170,55,241}

\newenvironment{problem}[2][Problem]{\begin{trivlist}
\item[\hskip \labelsep {\bfseries #1}\hskip \labelsep {\bfseries #2.}]}{\end{trivlist}}
%If you want to title your bold things something different just make another thing exactly like this but replace "problem" with the name of the thing you want, like theorem or lemma or whatever

\begin{document}
\lstset{language=Matlab,%
    %basicstyle=\color{red},
    breaklines=true,%
    morekeywords={matlab2tikz},
    keywordstyle=\color{blue},%
    morekeywords=[2]{1}, keywordstyle=[2]{\color{black}},
    identifierstyle=\color{black},%
    stringstyle=\color{mylilas},
    commentstyle=\color{mygreen},%
    showstringspaces=false,%without this there will be a symbol in the places where there is a space
    numbers=left,%
    numberstyle={\tiny \color{black}},% size of the numbers
    numbersep=9pt, % this defines how far the numbers are from the text
    emph=[1]{for,end,break},emphstyle=[1]\color{red}, %some words to emphasise
    %emph=[2]{word1,word2}, emphstyle=[2]{style},
}





\title{Econ 675: HW 2}
\author{Erin Markiewitz}
\maketitle
\newpage
\tableofcontents


\section{Kernel Density Estimation}
\subsection{}


First we consider the kernel density derivative estimator. $\hat{f}(x) =  \frac{1}{nh} \sum\limits_{i=1}^N k \left( \frac{X_i-x}{h} \right)$

The expectation of the estimator is:


\begin{gather*}
  \E[\hat{f}^{(s)}(x)] = \E[\hat{f}^{(s)}(x,h_n)] = \int\limits_{-\infty}^{\infty} \frac{(-1)^{s}}{h^{1+s}} k^{(s)} \left( \frac{z-x}{h} \right) f(z) dz
\end{gather*}


where $k^{(s)}$ is the $s^{th}$ derivative of the kernel function
Now integrate by parts

\begin{gather*}
\int\limits_{-\infty}^{\infty} \frac{(-1)^{s}}{h^{1+s}} k^{(s)} \left( \frac{z-x}{h} \right) f(z) dz =\\
 (-h)k^{(s-1)} \left( \frac{z-x}{h} \right)  f^{(1)}(z) |^\infty_{-\infty} - \int\limits_{-\infty}^{\infty}  \frac{(-1)^{s-1}}{h^{1+s-1}} k^{(s)} \left( \frac{z-x}{h} \right) f^{(1)}(z) dz
\end{gather*}


As $k(.)$ is a $P^{th}$ order kernel function and $s-1<P$, the first term on the RHS of the equation above is equal to zero. Integrating by parts s-1 more times and changing the base, we get the following expression

\begin{gather*}
\int\limits_{-\infty}^{\infty} k \left( u \right) f^{(s)}(uh+x) du
\end{gather*}


So now we take a $P^{th}$ order taylor expansion of $f^{(s)}(uh+x)$ around x, which gives us


\begin{gather*}
f^{(s)}(x) + \frac{1}{P!}\int\limits_{-\infty}^{\infty} k \left( u \right) f^{(s+P)}(uh+x) (uh + x - x)^p du + o(h_n^P)\\
= f^{(s)}(x) + \frac{1}{P!}\int\limits_{-\infty}^{\infty} k \left( u \right) f^{(s+P)}(uh+x) (uh)^p du + o(h_n^P) \\
= f^{(s)}(x) + \frac{f^{(s+P)}(x)}{P!} \mu_P(K) h_n^p +o(h_n^p)
\end{gather*}

where $ \mu_P(K) = \int_\R u^P K(u) du $ - which gives the result. (Note: the second term is the bias of the estimator)

\vpar

Now consider the variance of the estimator

\begin{gather*}
  \V[\hat{f}^{(s)}(x)] = \frac{1}{nh^{2+2s}} \V\left[[k^{(s)} \left( \frac{z-x}{h} \right)\right] = \frac{1}{nh^{2+2s}} \E\left[k^{(s)} \left( \frac{z-x}{h} \right)\right]^2 - \frac{1}{n} \E\left[\frac{1}{nh^{1+s}} k^{(s)} \left( \frac{z-x}{h} \right)\right]^2
\end{gather*}

Now using our derivation of the expected value of our estimator we can rewrite the expression above as:
\begin{gather*}
\frac{1}{nh^{2+2s}} \E\left[k^{(s)} \left( \frac{z-x}{h} \right)\right]^2 - \frac{1}{n}f^{(s)}(x)^2 + O\left(\frac{1}{n}\right)
\end{gather*}

(This comes from $\{ \frac{f^{(s+P)}(x)}{P!} \mu_P(K) h_n^p +o(h_n^p)\}$ being bounded)

So continuing on, we just expand the first term a bit

\begin{gather*}
\V[\hat{f}^{(s)}(x)] = \frac{1}{nh^{2+2s}} \int\limits_{-\infty}^{\infty} k^{(s)} \left( \frac{z-x}{h} \right)^2 f(z) dz - \frac{1}{n}f^{(s)}(x)^2 + O\left(\frac{1}{n}\right) \\
 = \frac{1}{nh^{1+2s}}  \int\limits_{-\infty}^{\infty} k^{(s)} \left( u \right) f(uh+x) du  - \frac{1}{n}f^{(s)}(x)^2 + O\left(\frac{1}{n}\right)  \\
 = \frac{f(x)}{nh^{1+2s}}  \int\limits_{-\infty}^{\infty} k^{(s)} \left( u \right) du  - \frac{1}{n}f^{(s)}(x)^2 + O\left(\frac{1}{n}\right)  \\
  = \frac{f(x) \nu_s(k) }{nh^{1+2s}}- \frac{1}{n}f^{(s)}(x)^2 + O\left(\frac{1}{n}\right)  \\
\end{gather*}

where $\nu_s(k) = \int\limits_\R k^{(s)} \left( u \right)^2 du $ is the roughness of the $s^{th}$ derivative of a given function $k$ - which gives the result.

\subsection{}

The optimal bandwith estimator solves the following problem
\begin{gather*}
\text{min}_h \text{   } AIMSE[h] = \text{min}_h \text{   } \int\limits_{-\infty}^{\infty} \left[ \left( h_n^p \mu_p(k) \frac{f^{(P+s)}(x)}{P!} \right)^2 + \frac{\nu_s(k) f(x)}{nh_n^{1+2s}} \right] dx
\end{gather*}

Take first order conditions

\begin{gather*}
0 =  2P h^{2P-1} \int\limits_{-\infty}^{\infty} \left[ \left( \mu_p(k) \frac{f^{(P+s)}(x)}{P!} \right)^2 - \frac{(1+2s) \nu_s(k) f(x)}{nh^{2s}} \right] dx\\
\end{gather*}


\begin{gather*}
\frac{2Pnh^{1-2P-2s}}{(1+2s) \nu_s(k)} = \left( \frac{P!}{\mu_p(k) \nu_{(P+s)}(f))} \right)^2\\
h_{AIMSE,s} = \left( \frac{ (1+2s) \nu_s(k) (P!)^2     }{ 2Pn \mu_p(k)^2 \nu_{(P+s)}(f))} \right)^{\frac{1}{1-2P-2s}}\\
h_{AIMSE,s} = \left( \frac{(1+2s)  (P!)^2     }{2Pn}     \frac{\nu_s(k)}{  \mu_p(k)^2 \nu_{(P+s)}(f))} \right)^{\frac{1}{1-2P-2s}}\\
\end{gather*}


Now for a consistent bandwith estimator we use cross validation procedure from the lecture notes.
\begin{gather*}
\hat{h}_{AIMSE,s} = \left( \frac{(1+2s)  (P!)^2     }{2Pn}     \frac{\hat{\nu}_s(k)}{  \hat{\mu}_p(k)^2 \hat{\nu}_{(P+s)}(f))} \right)^{\frac{1}{1-2P-2s}}\\
\end{gather*}

where
\begin{gather*}
\hat{\nu}_s(k) = \frac{1}{n}\sum\limits_{i=1}^N k^{(s)}(X_i)^2
\end{gather*}
?????

\newpage
\section{Linear Smoothing, Cross-Validation and Series}
\subsection{}

Local polynomial regression solves the following problem:

\begin{gather*}
\hat{\beta} =  \text{argmin}_{\beta \in \R^P+1} \text{    } \frac{1}{n}\sum\limits_{i=1}^N \left(Y_i - r_p(x-x_0)\beta  \right)^2 K(\frac{x_i - x}{h})
\end{gather*}

where $ r_p(u) =(1,u,u^2,...,u^p)' $ which can be rewritten as a weighted least-squares problem where $ \hat{\beta}(x) = \mathbf{ (R_p' W R_p)^{-1} R_p'WY }$  where the weighting matrix is a diagonal matrix with the kernel functions of the $ x_i $ s  the derivative kernel estimators s.t.

note $\hat{e}(x) = \frac{\hat{m}(x)}{\hat{f}(x)}$

$ WY = \sum\limits_{i=1}^N K(\frac{x_i - x}{h}) y_i$


Now we consider the series estimator, which solves the following problem
\begin{gather*}
\hat{\beta} =  \text{argmin}_{\beta \in \R^{k_n}} \text{    } \frac{1}{n}\sum\limits_{i=1}^N \left(Y_i - r_{k_n}(x)\beta  \right)^2 K(\frac{x_i - x}{h})
\end{gather*}

where $r_{k_n}(x)$ is the basis of some series defined on x, so that

\begin{gather*}
\hat{e}(x) = r_{k_n}(x)'\hat{\beta} =
\end{gather*}


\newpage
\section{Semiparametric Semi-Linear Model}
\subsection{}

The following question concerns this moment condition:
\begin{gather*}
\E[(t_i - h_0(x_i))(y_i - t_i\theta)] = 0 \text{  , where }  h_0(x_i) = \E[t_i | x_i]
\end{gather*}

As long as $t_i$ is not collinear with $x_i$ then $\theta_0$ will be identifiable. Assuming that $\theta_0$ is identifiable, it satisfies the moment condition above:

\begin{gather*}
\E[t_i y_i] + \E[h_0(x_i)t_i\theta] - \E[h_0(x_i)y_i] - \E[t_i t_i\theta)] = 0 \\
\E[\E[t_i y_i |t_i, x_i]] + \E[\E[h_0(x_i)t_i\theta|t_i, x_i]]  - \E[\E[h_0(x_i)y_i |t_i, x_i]]  + \E[\E[t_it_i\theta) |t_i, x_i]]  = 0 \\
\E[h_0(x_i) \E[y_i |t_i, x_i]] + \E[h_0(x_i)h_0(x_i)]\theta  - \E[h_0(x_i) \E[y_i |t_i, x_i]]  + \E[h_0(x_i)h_0(x_i)]\theta   = 0 \\
0=0
\end{gather*}

To derive a closed form equation for $\theta_0$ we follow the steps outlined in Hansen's notes on nonparametrics (chapter 7), which describes Robinson (Econometrica, 1988).

\begin{gather*}
y_i = t_i\theta_0 + g(x_i) + \epsilon_i
\end{gather*}

First we take the conditional expectation with respect to the treatment and other covariates. (We assume the treatment is not collinear with the other covariates.)

\begin{gather*}
\E[y_i|t_i x_i] = \E[t_i|t_i x_i]\theta_0 + \E[g(x_i)|t_i x_i]  + 0
\E[y_i|t_i x_i] = h_o(x_i)\theta_0 + g(x_i) + 0
\end{gather*}

Next, let's define $ g_{y,x}  \coloneqq \E[y_i|t_i x_i]$, and subtract the equation above from the original regression.


\begin{gather*}
y_i - g_{y,x} = (t_i -  h_o(x_i))\theta_0 + g(x_i) - g(x_i) + \epsilon_i
\end{gather*}

Now, we can rewrite the regression as a residual regression:

\begin{gather*}
\epsilon_{yi} =  \epsilon_{ti}\theta_0 + \epsilon_i\\
y_i = g_{y,x} + \epsilon_{yi}\\
t_i =  h_o(x_i) + \epsilon_{ti}
\end{gather*}

Which produces the infeasible estimtor:

\begin{gather*}
\beta = \left( \sum\limits_{i=1}^n \epsilon_{ti} \epsilon_{ti}' \right)^{-1} \left( \sum\limits_{i=1}^n \epsilon_{ti} \epsilon_{yi}' \right)
\end{gather*}

Note that we can rewrite the residual regression as :

\begin{gather*}
M_{yx} y_i = M_{tx} t_i \theta_0 + \epsilon_i
\end{gather*}

Which is the second stage of an IV regression that partials out the effects of $X_i$ on $y_i$ and $t_i$ using anhilation matrixes.

\subsection{}
\subsubsection{}

If the treatment is undetermined by the power series of the covariates, $\theta_0$ is simply

\begin{gather*}
\theta_0 = (T'T)^{-1}(T'Y)
\end{gather*}

which has a feasible estimator of

\begin{gather*}
\hat{\theta}(K) =(\sum\limits_{i=1}^{n} t_i t_i)^{-1}  (\sum\limits_{i=1}^{n} t_i y_i)\\
\end{gather*}










\subsubsection{}
If the treatment is correlated to the other covariates, in order to estimate a feasible estimator, one must run Nadaraya - Watson kernel regressions of the outcome and treatment variables onto the power series.

\begin{gather*}
\hat{y}_i = \frac{\sum\limits_{i=1}^{n} k\left(\frac{p^{K_n}(x_i) - p^{K_n}(x)}{h}\right) y_i}{\sum\limits_{i=1}^{n} k\left(\frac{p^{K_n}(x_i) - p^{K_n}(x)}{h}\right)} \\
h_0(x_i) = \frac{\sum\limits_{i=1}^{n} k\left(\frac{p^{K_n}(x_i) - p^{K_n}(x)}{h}\right) t_i}{\sum\limits_{i=1}^{n} k\left(\frac{p^{K_n}(x_i) - p^{K_n}(x)}{h}\right)}
\end{gather*}

Now, construct residualize
\begin{gather*}
\hat{\epsilon}_{yi} = y_i - \hat{y}_i = M_{yx}y_i \\
\hat{\epsilon}_{ti} = t_i - h_0(x_i)   = M_{tx}t_i
\end{gather*}

Which produces the feasible estimator

\begin{gather*}
\hat{\theta}(K) = \left( \sum\limits_{i=1}^n \hat{\epsilon}_{ti} \hat{\epsilon}_{ti}' \right)^{-1} \left( \sum\limits_{i=1}^n \hat{\epsilon}_{ti} \hat{\epsilon}_{yi}' \right)
\end{gather*}


\subsection{}
\subsubsection{}
Fixing K, the reason this approach is called a "flexible parametric" estimation because you are estimating $\theta_0$, while letting

If $K\rightarrow\infty$ does not invalidate the "fixed K" assumption as long as the ratio between the observations and covariates is fixed $\left( \frac{K_n}{n} = \frac{\bar{K}}{\bar{n}}   \right)$

\subsubsection{}
Using the results above the confidence interval is

\begin{gather*}
CI_{95} = \left[\hat{\theta}(K)  - 1.96  \sqrt{\hat{V}_{HCO} / n}  ; \hat{\theta}(K) + 1.96  \sqrt{\hat{V}_{HCO} / n} \right]
\end{gather*}

\subsection{}


\end{document}
